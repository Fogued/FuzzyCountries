\begin{titlepage}

\begin{minipage}{0.15\linewidth}
\hspace*{-15mm}
\noindent
\includegraphics[scale=0.5]{front-page/escudo_upm.png}
\end{minipage}
\begin{minipage}{0.7\linewidth}
\begin{center}
\huge{ Universidad Politécnica\\de Madrid }\\
\vspace*{0.5cm}
\Large{\textbf{Escuela Técnica Superior de \\
Ingenieros Informáticos}}
\end{center}
\end{minipage}
\begin{minipage}{0.2\linewidth}
\includegraphics[scale=0.5]{front-page/escudo_etsiinf.png}
\end{minipage}

\vspace*{1cm}
\begin{center}
\Large{Grado Matemáticas e Informática}
\end{center}

\vspace*{2.5cm}
\begin{center}
\huge\bfseries {Fuzzy Countries}
\end{center}

\vspace*{2cm}
\begin{center}
\textit{In this project a socioeconomical model of a variety of countries is developed. 
\\The model is based on the fuzzy logic theory and it is implemented in Ciao Prolog using the RFuzzy library and Python using Sklearn to compare the results with a real dataset in order to find the credibility of the model. For the visualization of the results, Uflese is used.} 
\end{center}

\vspace*{3cm}
\noindent
\large{Authors: Javier Comyn, Diego Fogued, Francisco J. González}\\
\large{Professor: Susana Muñoz Hernández}

\vspace*{2cm}
\begin{center}
Madrid, 2023/2024
\end{center}
\end{titlepage}
