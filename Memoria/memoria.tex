\documentclass[fleqn,11pt]{article}
\usepackage[utf8]{inputenc}
\usepackage[english]{babel}
\usepackage{multicol}
\usepackage{graphicx}
\usepackage{amsfonts,amsmath,amssymb}
\usepackage{enumitem}
\usepackage{booktabs}
\usepackage{colortbl}
\newcommand{\grad}{$^{\circ}$}
\usepackage{subcaption}
\usepackage{multirow}
\usepackage{array}
\usepackage{bigints}
\usepackage{rotating}
\usepackage{xcolor}
\usepackage{inputenc}
\usepackage{wrapfig}
\usepackage{cancel}
\usepackage{titlesec}
\usepackage{mathtools}
\setcounter{secnumdepth}{4}
\renewcommand{\baselinestretch}{1.2}
\usepackage{hyperref}
\hypersetup{
	colorlinks=true,
	linkcolor=black,
	filecolor=magenta,
	urlcolor=blue,
	citecolor=green,
	pdfpagemode=FullScreen
}
\makeindex

\usepackage[a4paper,textheight=24cm,textwidth=16cm]{geometry}

\setlength{\parindent}{0cm}
\setlength{\parskip}{10pt}
\setlength{\mathindent}{1cm}
\pagestyle{headings}
\begin{document}
\begin{titlepage}

\begin{minipage}{0.15\linewidth}
\hspace*{-15mm}
\noindent
\includegraphics[scale=0.5]{front-page/escudo_upm.png}
\end{minipage}
\begin{minipage}{0.7\linewidth}
\begin{center}
\huge{ Universidad Politécnica\\de Madrid }\\
\vspace*{0.5cm}
\Large{\textbf{Escuela Técnica Superior de \\
Ingenieros Informáticos}}
\end{center}
\end{minipage}
\begin{minipage}{0.2\linewidth}
\includegraphics[scale=0.5]{front-page/escudo_etsiinf.png}
\end{minipage}

\vspace*{1cm}
\begin{center}
\Large{Grado Matemáticas e Informática}
\end{center}

\vspace*{2.5cm}
\begin{center}
\huge\bfseries {Fuzzy Countries}
\end{center}

\vspace*{2cm}
\begin{center}
\textit{In this project, a socioeconomic model for various countries is developed using fuzzy logic.
\\The model is implemented in Ciao Prolog with the RFuzzy library and using Python with Scikit-learn to compare the results with real data and assess the model's credibility. Uflese is used for visualizing the results.} 
\end{center}

\vspace*{3cm}
\noindent
\large{Authors: Javier Comyn, Diego Fogued, Francisco J. González}\\
\large{Professor: Susana Muñoz Hernández}

\vspace*{2cm}
\begin{center}
Madrid, 2023/2024
\end{center}
\end{titlepage}


\newpage
\tableofcontents

\newpage

\section{Introduction}

\subsection{Background and motivation for the study}
The motivation for this study comes from the need to better understand and model the complex socioeconomic dynamics of different countries.
Traditional economic models often can't handle the uncertainty and vagueness in real-world data.
Fuzzy logic theory is well-suited for this task, providing a way to deal with these uncertainties.
This study aims to create a more accurate and reliable socioeconomic model, which will be compared with real data to ensure its credibility.

\subsection{Research objectives}
The main objective of this research is to develop a socioeconomic model that gives relevant insights into the economic and environmental conditions of different countries, which would not be possible with traditional models and classical logic
Additionally, the research seeks to use Uflese for visualizing the outcomes, ensuring that the model's findings are both understandable and useful for further analysis. 
Ultimately, the goal is to establish a credible model that can provide valuable insights into the socioeconomic conditions of various countries.

\section{Theorical Framework}

\subsection{Fuzzy Logic}
Fuzzy logic is a form of many-valued logic in which the truth values of variables may be any real number between 0 and 1 both inclusive. It is employed to handle the concept of partial truth, where the truth value may range between completely true and completely false. By contrast, in Boolean logic, the truth values of variables may only be the integer values 0 or 1.

Fuzzy logic has been extended to handle the concept of partial truth, where the truth value may range between completely true and completely false. Furthermore, when linguistic variables are used, these degrees may be managed by specific functions.

\section{Methodology}



\section{Database Design and Development}

%Description of the database schema.
%Explanation of the data types used (rfuzzy_string_type, rfuzzy_float_type, etc.).
%Data sources and data integration.
%Validation of the database.

\subsection{Data Collection}
To gather the data, we used a variety of sources (mainly Kaggle) to obtain information on different socio-economic and environmental indicators for various countries.
We analysed which indicators would be most relevant for our study and selected the most reliable and up-to-date datasets available.
Furthermore, we ensured that the data was clean and consistent by performing data cleaning and validation procedures.

\subsection{Data Description}
The variables in the database include a mix of socio-economic and environmental indicators, which are:
\begin{itemize}  %TODO: COMPLETAR Y EXPLPICAR QUE SON CADA UNA DE LAS VARIABLES Y SUS UNIDADES
	\item GDP per capita
	\item Renewable energy consumption
	\item Minimum wage
	\item Population density
	\item Economic freedom index
	\item Mean temperature
	\item Life expectancy
	\item Happiness index

\subsection{Data Preprocessing}
Before integrating the data into the database, we performed several preprocessing steps to clean and transform the data.
This included handling missing values, normalizing the data, and converting categorical variables into numerical values.
Additionaly, the different datasets were merged and integrated into a single database, ensuring that the data was consistent and ready for analysis.

\subsection{Data Analysis}


\section{Implementation of the Fuzzy System}

%Development environment and tools used.
%Detailed explanation of the implementation process.
%Definition of the fuzzy rules for different socio-economic and environmental indicators.
%Integration of the fuzzy logic system with the database.

\section{Results and Discussion}

%Presentation of the results obtained from the fuzzy logic system.
%Analysis of the results in the context of socio-economic and environmental indicators.
%Comparison with traditional methods.
%Case studies or examples.

\subsection{Querys}

\section{Challenges and Solutions}

\subsection{Identified Difficulties}

%Data Collection and Quality: Issues related to data availability, accuracy, and completeness.
%Problems with Data Preprocessing: Challenges in cleaning, transforming, and integrating data from multiple sources.
%Fuzzy Logic System Design: Difficulties in defining appropriate membership functions and rules.
%Technical Implementation: Problems encountered during the coding and integration of the fuzzy logic system.
%Interpretation of Results: Challenges in making sense of the fuzzy logic output and its implications.


\subsection{Overcoming Difficulties}

%Strategies for obtaining and validating reliable data.
%Iterative refinement of fuzzy logic rules and membership functions based on expert feedback and testing.
%Use of robust development tools and platforms to ensure smooth technical implementation.
%Collaboration with domain experts to enhance the interpretation and applicability of results.


\section{Conclusions and Future Work}

\newpage
\begin{thebibliography}{10}
	
\bibitem{1} \textsc{Elgiriyewithana, Nidula}, \textit{Global Country Information Dataset 2023.} [Data set]. (2023, 8 julio). Kaggle. \newline
\small{\url{https://www.kaggle.com/datasets/nelgiriyewithana/countries-of-the-world-2023}}

\bibitem{2} \textsc{Hossainds, Belayet}, \textit{Renewable Energy world Wide : 1965~2022.} [Data set]. (2023, 3 marzo). Kaggle. \newline
\small{\url{https://www.kaggle.com/datasets/belayethossainds/renewable-energy-world-wide-19652022}}

\bibitem{3} \textsc{Pei Pei Chen}, \textit{Minimum wage by country.} [Data set]. (2020, 27 diciembre). Kaggle. \newline
\small{\url{https://www.kaggle.com/datasets/peipeichen/minimum-wage-by-country}}

\bibitem{4} \textsc{My Koryto}, \textit{countryinfo.} [Data set]. (2020, 14 abril). Kaggle. \newline
\small{\url{https://www.kaggle.com/datasets/koryto/countryinfo}}

\bibitem{5} \textsc{Fraser Institute}, \textit{Economic Freedom of the World.} [Data set]. (2021). \newline
\small{\url{https://www.fraserinstitute.org/economic-freedom/dataset?geozone=world&min-year=2&max-year=0&filter=0&page=dataset&year=2021}}

\bibitem{6} \textsc{Palinatx}, \textit{Mean temperature for countries by year 1901-2022.} [Data set]. (2024, 21 marzo). Kaggle. \newline
\small{\url{https://www.kaggle.com/datasets/palinatx/mean-temperature-for-countries-by-year-2014-2022/suggestions?status=pending&yourSuggestions=true}}

\bibitem{7} \textsc{World Health Organization}, \textit{Indicators}. [Data set]. (n.d.). World Health Organization. \newline
\small{\url{https://data.who.int/es/indicators}}

\bibitem{8} \textsc{Beach, J.}, \textit{World Happiness Report 2013-2023} [Data set]. (2023). Kaggle. \newline
\small{\url{https://www.kaggle.com/datasets/joebeachcapital/world-happiness-report-2013-2023}}

\bibitem{Singh2021} \textsc{Singh, A. P.}, \textit{World Happiness Report 2021} [Notebook]. (2021). Kaggle. \newline
\small{\url{https://www.kaggle.com/code/ajaypalsinghlo/world-happiness-report-2021-world/notebook}}

\end{thebibliography}



\end{document}
