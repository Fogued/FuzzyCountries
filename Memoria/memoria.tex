\documentclass[fleqn,11pt]{article}
\usepackage[utf8]{inputenc}
\usepackage[english]{babel}
\usepackage{multicol}
\usepackage{graphicx}
\usepackage{amsfonts,amsmath,amssymb}
\usepackage{enumitem}
\usepackage{booktabs}
\usepackage{colortbl}
\newcommand{\grad}{$^{\circ}$}
\usepackage{subcaption}
\usepackage{multirow}
\usepackage{array}
\usepackage{bigints}
\usepackage{rotating}
\usepackage{xcolor}
\usepackage{inputenc}
\usepackage{wrapfig}
\usepackage{cancel}
\usepackage{titlesec}
\usepackage{mathtools}
\setcounter{secnumdepth}{4}
\renewcommand{\baselinestretch}{1.2}
\usepackage{hyperref}
\hypersetup{
	colorlinks=true,
	linkcolor=black,
	filecolor=magenta,
	urlcolor=cyan,
	citecolor=green,
	pdfpagemode=FullScreen
}
\makeindex

\usepackage[a4paper,textheight=24cm,textwidth=16cm]{geometry}

\setlength{\parindent}{0cm}
\setlength{\parskip}{10pt}
\setlength{\mathindent}{1cm}
\pagestyle{headings}
\begin{document}
\input{portada/portada.tex}

\newpage
\tableofcontents

\newpage

\section{Theorical Framework}



\subsection{Fuzzy Logic}

Fuzzy logic is a form of many-valued logic in which the truth values of variables may be any real number between 0 and 1 both inclusive. It is employed to handle the concept of partial truth, where the truth value may range between completely true and completely false. By contrast, in Boolean logic, the truth values of variables may only be the integer values 0 or 1.


\end{document}
