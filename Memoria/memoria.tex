\documentclass[fleqn,11pt]{article}
\usepackage[utf8]{inputenc}
\usepackage[english]{babel}
\usepackage{multicol}
\usepackage{graphicx}
\usepackage{amsfonts,amsmath,amssymb}
\usepackage{enumitem}
\usepackage{booktabs}
\usepackage{colortbl}
\newcommand{\grad}{$^{\circ}$}
\usepackage{subcaption}
\usepackage{multirow}
\usepackage{array}
\usepackage{bigints}
\usepackage{rotating}
\usepackage{xcolor}
\usepackage{inputenc}
\usepackage{wrapfig}
\usepackage{cancel}
\usepackage{titlesec}
\usepackage{mathtools}
\setcounter{secnumdepth}{4}
\renewcommand{\baselinestretch}{1.2}
\usepackage{hyperref}
\hypersetup{
	colorlinks=true,
	linkcolor=black,
	filecolor=magenta,
	urlcolor=cyan,
	citecolor=green,
	pdfpagemode=FullScreen
}
\makeindex

\usepackage[a4paper,textheight=24cm,textwidth=16cm]{geometry}

\setlength{\parindent}{0cm}
\setlength{\parskip}{10pt}
\setlength{\mathindent}{1cm}
\pagestyle{headings}
\begin{document}
\begin{titlepage}

\begin{minipage}{0.15\linewidth}
\hspace*{-15mm}
\noindent
\includegraphics[scale=0.5]{front-page/escudo_upm.png}
\end{minipage}
\begin{minipage}{0.7\linewidth}
\begin{center}
\huge{ Universidad Politécnica\\de Madrid }\\
\vspace*{0.5cm}
\Large{\textbf{Escuela Técnica Superior de \\
Ingenieros Informáticos}}
\end{center}
\end{minipage}
\begin{minipage}{0.2\linewidth}
\includegraphics[scale=0.5]{front-page/escudo_etsiinf.png}
\end{minipage}

\vspace*{1cm}
\begin{center}
\Large{Grado Matemáticas e Informática}
\end{center}

\vspace*{2.5cm}
\begin{center}
\huge\bfseries {Fuzzy Countries}
\end{center}

\vspace*{2cm}
\begin{center}
\textit{In this project, a socioeconomic model for various countries is developed using fuzzy logic.
\\The model is implemented in Ciao Prolog with the RFuzzy library and using Python with Scikit-learn to compare the results with real data and assess the model's credibility. Uflese is used for visualizing the results.} 
\end{center}

\vspace*{3cm}
\noindent
\large{Authors: Javier Comyn, Diego Fogued, Francisco J. González}\\
\large{Professor: Susana Muñoz Hernández}

\vspace*{2cm}
\begin{center}
Madrid, 2023/2024
\end{center}
\end{titlepage}


\newpage
\tableofcontents

\newpage

\section{Introduction}

\subsection{Background and motivation for the study}

\subsection{Research objectives}

\section{Theorical Framework}

\subsection{Fuzzy Logic}
Fuzzy logic is a form of many-valued logic in which the truth values of variables may be any real number between 0 and 1 both inclusive. It is employed to handle the concept of partial truth, where the truth value may range between completely true and completely false. By contrast, in Boolean logic, the truth values of variables may only be the integer values 0 or 1.

Fuzzy logic has been extended to handle the concept of partial truth, where the truth value may range between completely true and completely false. Furthermore, when linguistic variables are used, these degrees may be managed by specific functions.

\section{Methodology}



\section{Database Design and Development}

%Description of the database schema.
%Explanation of the data types used (rfuzzy_string_type, rfuzzy_float_type, etc.).
%Data sources and data integration.
%Validation of the database.

\subsection{Data Collection}

\subsection{Data Description}

\subsection{Data Analysis}

\subsection{Data Preprocessing}

\section{Implementation of the Fuzzy System}

%Development environment and tools used.
%Detailed explanation of the implementation process.
%Definition of the fuzzy rules for different socio-economic and environmental indicators.
%Integration of the fuzzy logic system with the database.

\section{Results and Discussion}

%Presentation of the results obtained from the fuzzy logic system.
%Analysis of the results in the context of socio-economic and environmental indicators.
%Comparison with traditional methods.
%Case studies or examples.

\subsection{Querys}

\section{Challenges and Solutions}

\subsection{Identified Difficulties}

%Data Collection and Quality: Issues related to data availability, accuracy, and completeness.
%Problems with Data Preprocessing: Challenges in cleaning, transforming, and integrating data from multiple sources.
%Fuzzy Logic System Design: Difficulties in defining appropriate membership functions and rules.
%Technical Implementation: Problems encountered during the coding and integration of the fuzzy logic system.
%Interpretation of Results: Challenges in making sense of the fuzzy logic output and its implications.


\subsection{Overcoming Difficulties}

%Strategies for obtaining and validating reliable data.
%Iterative refinement of fuzzy logic rules and membership functions based on expert feedback and testing.
%Use of robust development tools and platforms to ensure smooth technical implementation.
%Collaboration with domain experts to enhance the interpretation and applicability of results.


\section{Conclusions and Future Work}



\end{document}
