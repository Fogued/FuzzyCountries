\documentclass[fleqn,11pt]{article}
\usepackage[utf8]{inputenc}
\usepackage[english]{babel}
\usepackage{multicol}
\usepackage{graphicx}
\usepackage{amsfonts,amsmath,amssymb}
\usepackage{enumitem}
\usepackage{booktabs}
\usepackage{colortbl}
\newcommand{\grad}{$^{\circ}$}
\usepackage{subcaption}
\usepackage{multirow}
\usepackage{array}
\usepackage{bigints}
\usepackage{rotating}
\usepackage{xcolor}
\usepackage{inputenc}
\usepackage{wrapfig}
\usepackage{cancel}
\usepackage{titlesec}
\usepackage{mathtools}
\setcounter{secnumdepth}{4}
\renewcommand{\baselinestretch}{1.2}
\usepackage{hyperref}
\hypersetup{
	colorlinks=true,
	linkcolor=black,
	filecolor=magenta,
	urlcolor=blue,
	citecolor=green,
	pdfpagemode=FullScreen
}
\makeindex

\usepackage[a4paper,textheight=24cm,textwidth=16cm]{geometry}

\setlength{\parindent}{0cm}
\setlength{\parskip}{10pt}
\setlength{\mathindent}{1cm}
\pagestyle{headings}
\begin{document}
\begin{titlepage}

\begin{minipage}{0.15\linewidth}
\hspace*{-15mm}
\noindent
\includegraphics[scale=0.5]{front-page/escudo_upm.png}
\end{minipage}
\begin{minipage}{0.7\linewidth}
\begin{center}
\huge{ Universidad Politécnica\\de Madrid }\\
\vspace*{0.5cm}
\Large{\textbf{Escuela Técnica Superior de \\
Ingenieros Informáticos}}
\end{center}
\end{minipage}
\begin{minipage}{0.2\linewidth}
\includegraphics[scale=0.5]{front-page/escudo_etsiinf.png}
\end{minipage}

\vspace*{1cm}
\begin{center}
\Large{Grado Matemáticas e Informática}
\end{center}

\vspace*{2.5cm}
\begin{center}
\huge\bfseries {Fuzzy Countries}
\end{center}

\vspace*{2cm}
\begin{center}
\textit{In this project, a socioeconomic model for various countries is developed using fuzzy logic.
\\The model is implemented in Ciao Prolog with the RFuzzy library and using Python with Scikit-learn to compare the results with real data and assess the model's credibility. Uflese is used for visualizing the results.} 
\end{center}

\vspace*{3cm}
\noindent
\large{Authors: Javier Comyn, Diego Fogued, Francisco J. González}\\
\large{Professor: Susana Muñoz Hernández}

\vspace*{2cm}
\begin{center}
Madrid, 2023/2024
\end{center}
\end{titlepage}


\newpage
\tableofcontents

\newpage

\section{Introduction}

\subsection{Background and motivation for the study}

At the beginning, when we started thinking about the project, we were looking for a topic that could be both fascinating and challenging, while also fitting well with the principles of fuzzy logic. We aimed to choose a topic applicable to real life, allowing us to draw conclusions that we might not have realized without applying these tools.

At the very first moment, we started thinking about the possibility of focusing on psychological analysis or something related to human mental health because we thought it would be interesting to apply fuzzy logic to this field. However, we quickly realized that this topic was too broad and complex for the scope of our project and also that it would be difficult to find reliable data to work with.

Without giving up on the idea of working with human behavior, we decided to focus on a topic that would allow us to analyze human behavior in a more indirect way. We thought about the possibility of analyzing the relationship between socio-economic and environmental indicators and how these factors can influence the happiness of a country's population. We believe that this topic is relevant and interesting because it allows us to explore the relationship between different aspects of human life and how they can affect people's well-being.

Moreover, this idea of analyzing the happiness of a country's population gives us the opportunity to contrast the results obtained with the World Happiness Report, which is a well-known study that ranks countries based on their happiness levels. This will allow us to validate our results and compare them with those obtained by other researchers, thereby assessing the credibility of this approach.
\subsection{Research objectives}

As mentioned above, the main objective of this project is to analyze the relationship between socio-economic and environmental indicators and the happiness of a country's population. To achieve this, we will develop a fuzzy logic system with functions and rules to model this relationship and draw conclusions from the available data.

We will use data from reputable sources like the World Happiness Report and the World Bank. Our fuzzy logic system, with functions and rules will process this data to reveal patterns and insights that may not be immediately apparent, and we will find the credibility of our results by comparing the happiness scores we obtain with those in the World Happiness Report.


\section{Theorical Framework}

\subsection{Fuzzy Logic}
Fuzzy logic is a form of many-valued logic in which the truth values of variables may be any real number between 0 and 1 both inclusive. It is employed to handle the concept of partial truth, where the truth value may range between completely true and completely false. By contrast, in Boolean logic, the truth values of variables may only be the integer values 0 or 1.

Fuzzy logic has been extended to handle the concept of partial truth, where the truth value may range between completely true and completely false. Furthermore, when linguistic variables are used, these degrees may be managed by specific functions.

\section{Methodology}

The methodology used in this project can be divided into the following steps:

\begin{enumerate}
	\item Data Collection: Collecting data from different sources related to socio-economic and environmental indicators.
	\item Data Description: Describing the data collected and analyzing its characteristics.
	\item Data Preprocessing: Cleaning, transforming, and integrating the data to make it suitable for analysis.
	\item Database Design and Development: Designing and developing a database to store the data and integrate it with the fuzzy logic system.
	\item Implementation of the Fuzzy System: Developing the fuzzy logic system with functions and rules to model the relationship between the indicators and happiness.
	\item Results and Discussion: Presenting and analyzing the results obtained from the fuzzy logic system.
	\item Challenges and Solutions: Identifying difficulties encountered during the project and proposing solutions to overcome them.
	\item Conclusions and Future Work: Drawing conclusions from the study and suggesting possible future research directions.
\end{enumerate}



\section{Database Design and Development}

%Description of the database schema.
%Explanation of the data types used (rfuzzy_string_type, rfuzzy_float_type, etc.).
%Data sources and data integration.
%Validation of the database.

\subsection{Data Collection}

First of all, we started by collecting data from different sources. 

\subsection{Data Description}

\subsection{Data Analysis}

\subsection{Data Preprocessing}

\section{Implementation of the Fuzzy System}

%Development environment and tools used.
%Detailed explanation of the implementation process.
%Definition of the fuzzy rules for different socio-economic and environmental indicators.
%Integration of the fuzzy logic system with the database.

\section{Results and Discussion}

%Presentation of the results obtained from the fuzzy logic system.
%Analysis of the results in the context of socio-economic and environmental indicators.
%Comparison with traditional methods.
%Case studies or examples.

\subsection{Querys}

\section{Challenges and Solutions}

\subsection{Identified Difficulties}

%Data Collection and Quality: Issues related to data availability, accuracy, and completeness.
%Problems with Data Preprocessing: Challenges in cleaning, transforming, and integrating data from multiple sources.
%Fuzzy Logic System Design: Difficulties in defining appropriate membership functions and rules.
%Technical Implementation: Problems encountered during the coding and integration of the fuzzy logic system.
%Interpretation of Results: Challenges in making sense of the fuzzy logic output and its implications.


\subsection{Overcoming Difficulties}

%Strategies for obtaining and validating reliable data.
%Iterative refinement of fuzzy logic rules and membership functions based on expert feedback and testing.
%Use of robust development tools and platforms to ensure smooth technical implementation.
%Collaboration with domain experts to enhance the interpretation and applicability of results.


\section{Conclusions and Future Work}

\newpage
\begin{thebibliography}{10}
	
\bibitem{1} \textsc{Elgiriyewithana, Nidula}, \textit{Global Country Information Dataset 2023.} [Data set]. (2023, 8 julio). Kaggle. \newline
\small{\url{https://www.kaggle.com/datasets/nelgiriyewithana/countries-of-the-world-2023}}

\bibitem{2} \textsc{Hossainds, Belayet}, \textit{Renewable Energy world Wide : 1965~2022.} [Data set]. (2023, 3 marzo). Kaggle. \newline
\small{\url{https://www.kaggle.com/datasets/belayethossainds/renewable-energy-world-wide-19652022}}

\bibitem{3} \textsc{Pei Pei Chen}, \textit{Minimum wage by country.} [Data set]. (2020, 27 diciembre). Kaggle. \newline
\small{\url{https://www.kaggle.com/datasets/peipeichen/minimum-wage-by-country}}

\bibitem{4} \textsc{My Koryto}, \textit{countryinfo.} [Data set]. (2020, 14 abril). Kaggle. \newline
\small{\url{https://www.kaggle.com/datasets/koryto/countryinfo}}

\bibitem{5} \textsc{Fraser Institute}, \textit{Economic Freedom of the World.} [Data set]. (2021). \newline
\small{\url{https://www.fraserinstitute.org/economic-freedom/dataset?geozone=world&min-year=2&max-year=0&filter=0&page=dataset&year=2021}}

\bibitem{6} \textsc{Palinatx}, \textit{Mean temperature for countries by year 1901-2022.} [Data set]. (2024, 21 marzo). Kaggle. \newline
\small{\url{https://www.kaggle.com/datasets/palinatx/mean-temperature-for-countries-by-year-2014-2022/suggestions?status=pending&yourSuggestions=true}}

\bibitem{7} \textsc{World Health Organization}, \textit{Indicators}. [Data set]. (n.d.). World Health Organization. \newline
\small{\url{https://data.who.int/es/indicators}}

\bibitem{8} \textsc{Beach, J.}, \textit{World Happiness Report 2013-2023} [Data set]. (2023). Kaggle. \newline
\small{\url{https://www.kaggle.com/datasets/joebeachcapital/world-happiness-report-2013-2023}}

\bibitem{Singh2021} \textsc{Singh, A. P.}, \textit{World Happiness Report 2021} [Notebook]. (2021). Kaggle. \newline
\small{\url{https://www.kaggle.com/code/ajaypalsinghlo/world-happiness-report-2021-world/notebook}}

\end{thebibliography}



\end{document}
