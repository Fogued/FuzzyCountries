\documentclass{beamer}
\usetheme{Madrid}
\usepackage[spanish]{babel}

\usepackage{tikz}

% Esto crea la portada de cada section
\AtBeginSection[]{
  \begin{frame}[plain]
  \begin{tikzpicture}[remember picture,overlay]
    \fill[structure.fg] (current page.south west) rectangle (current page.north east);
    \node at (current page.center) {
\Huge\color{white}\textbf{\insertsection}
    };
  \end{tikzpicture}
  \end{frame}
}

\title{Fuzzy Countries}
\author{Javier Comyn, Diego Fogued y Francisco J. González}
\institute{Universidad Politécnica de Madrid}
\date{Curso 2023/2024}

% Define a command to set the author for each slide
\newcommand{\slideauthor}[1]{\def\insertslideauthor{#1}}
\newcommand{\insertslideauthor}{}

% Set color for footline
\setbeamercolor{author in head/foot}{fg=white, bg=blue}

% Customize the footline
\setbeamertemplate{footline}
{
    \leavevmode%
    \hbox{%
    \begin{beamercolorbox}[wd=.4\paperwidth,ht=2.25ex,dp=1ex,left]{author in head/foot}%
        \usebeamerfont{author in head/foot}\insertshortauthor\hspace*{2em}
    \end{beamercolorbox}%
    \begin{beamercolorbox}[wd=.6\paperwidth,ht=2.25ex,dp=1ex,right]{title in head/foot}%
        \usebeamerfont{title in head/foot}\insertshorttitle\hspace*{2em}
    \end{beamercolorbox}}%
    \vskip0pt%
}

\begin{document}
\frame{\titlepage}

\begin{frame}
\frametitle{Índice de contenidos}
\tableofcontents
\end{frame}

\section{Introducción}
\begin{frame}
\frametitle{Background y motivación}
Este es un texto en la primera diapositiva. Este es un texto en la primera diapositiva. Este es un texto en la primera diapositiva.
\end{frame}
\begin{frame}
\frametitle{Objetivos}
texto
\end{frame}

\section{Marco teórico}
\begin{frame}
\frametitle{Marco teórico}
\slideauthor{Autor 2}

\end{frame}

\section{Metodología}
\begin{frame}
\frametitle{Metodología}
\slideauthor{Autor 2}
texto
\end{frame}

\section{Diseño de la Base de Datos}
\begin{frame}
\frametitle{Recopilación}
\slideauthor{Autor 2}
\begin{itemize}
    \item Consultar fuentes: Banco Mundial, OMS, Kaggle...
    \item Escoger indicadores más relevantes para un análisis socioeconómico.
    \item Elección de los conjuntos de datos más confiables y actualizados.
    \item Asegurarse de la consistencia y veracidad.
\end{itemize}
\end{frame}
\begin{frame}
\frametitle{Descripción}
\slideauthor{Autor 2}
\begin{itemize}
    \item Índice de libertad económica
    \item Temperatura media (ºC)
    \item Tasa suicidios por 100.000 habitantes
    \item Percepción de la corrupción
    \item Densidad de población
    \item Porcentaje de terreno agrícola
    \item Superficie
    \item Tamaño del ejército
    \item Tasa de natalidad
    \item CO2
    \item Índice de Precios al Consumidor (IPC)
    \item Tasa de fertilidad
    \item Porcentaje de área forestal
\end{itemize}
\end{frame}
\begin{frame}
\frametitle{Descripción}
\slideauthor{Autor 2}
\begin{itemize}
    \item PIB per cápita
    \item Alumnos en educación primaria
    \item Alumnos en educación post-obligatoria
    \item Mortalidad infantil
    \item Esperanza de vida
    \item Tamaño de la población
    \item Población activa
    \item Ingresos fiscales (\% del PIB)
    \item Tasa de paro
    \item Población urbana
    \item Energías renovables
    \item Salario mínimo
    \item Edad media
\end{itemize}
\end{frame}
\begin{frame}
\frametitle{Preprocesamiento y limpieza}
\slideauthor{Autor 2}
\begin{itemize}
    \item Integrar todas las variables en una única base de datos
    \item Eliminar inconsistencias
    \item Tratar valores faltantes
    \item Convertir todos los valores en enteros
\end{itemize}
\end{frame}

\section{Análisis de datos}
\begin{frame}
\frametitle{Sistema de Lógica difusa}
\slideauthor{Autor 2}
Explicar functions y rules, mostar ejemplos...
\end{frame}
\begin{frame}
\frametitle{Consultas}
\slideauthor{Autor 2}
Fotos de consultas
\end{frame}
\begin{frame}
\frametitle{Resultados notables}
\slideauthor{Autor 2}
Fotos de consultas
\end{frame}

\section{Optimización}
\begin{frame}
\frametitle{Optimización}
\slideauthor{Autor 2}

\end{frame}

\section{Resultados y Conclusiones}
\begin{frame}
\frametitle{Resultados}
\slideauthor{Autor 2}
texto
\end{frame}

\end{document}
