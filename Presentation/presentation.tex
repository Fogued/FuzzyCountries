\documentclass{beamer}
\usetheme{Madrid}
\usepackage[spanish]{babel}

\title{Fuzzy Countries}
\author{Javier Comyn, Diego Fogued y Francisco J. González}
\institute{Universidad Polotécnica de Madrid}
\date{Curso 2023/2024}

% Define a command to set the author for each slide
\newcommand{\slideauthor}[1]{\def\insertslideauthor{#1}}
\newcommand{\insertslideauthor}{}

% Set color for footline
\setbeamercolor{author in head/foot}{fg=white, bg=blue}

% Customize the footline
\setbeamertemplate{footline}
{
    \leavevmode%
    \hbox{%
    \begin{beamercolorbox}[wd=.4\paperwidth,ht=2.25ex,dp=1ex,left]{author in head/foot}%
        \usebeamerfont{author in head/foot}\insertshortauthor\hspace*{2em}
    \end{beamercolorbox}%
    \begin{beamercolorbox}[wd=.6\paperwidth,ht=2.25ex,dp=1ex,right]{title in head/foot}%
        \usebeamerfont{title in head/foot}\insertshorttitle\hspace*{2em}
    \end{beamercolorbox}}%
    \vskip0pt%
}

\begin{document}

\frame{\titlepage}

\begin{frame}
\frametitle{Índice de contenidos}
\tableofcontents
\end{frame}

\section{Introducción}
\begin{frame}
\frametitle{Background y motivación}
Este es un texto en la primera diapositiva. Este es un texto en la primera diapositiva. Este es un texto en la primera diapositiva.
\end{frame}
\begin{frame}
\frametitle{Objetivos}
texto
\end{frame}
\section{Sección 2}
\begin{frame}
\frametitle{Otro título de diapositiva}
\slideauthor{Autor 2}
Este es un texto en la segunda diapositiva. Este es un texto en la segunda diapositiva. Este es un texto en la segunda diapositiva.
\end{frame}

\end{document}